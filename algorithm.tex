\documentclass{article}

\usepackage{geometry}
\usepackage{xeCJK}
\usepackage{minted}
\usepackage[colorlinks,linkcolor=blue]{hyperref}
\usepackage{amsmath}

% 设置页大小和页边距,或者scale=0.8
\geometry{a4paper,left=3.18cm,right=3.18cm,top=2.54cm,bottom=2.54cm}
% 声明分区,等宽字体:制表符号,等宽空格,ASCII,↺(循环箭头)
\xeCJKDeclareSubCJKBlock{MonoFont}{ "2500 -> "257F, "2002, "01 -> "F0, "21BA }
% 中文默认没有斜体和粗体格式,开启伪斜体和指定黑体;MonoFont分区指定字体为等宽字体Menlo
\setCJKmainfont[AutoFakeSlant, BoldFont=SimHei, MonoFont=Menlo]{SimSun}
% 全局设置代码样式,空格替换成英文等宽的字符0x2002
\setminted[java]{linenos, showspaces, space=\char"2002, fontsize=\small}

\begin{document}
\section{基础}
  \subsection{最大公约数}
  辗转相除法:
  \begin{enumerate}
    \item p和q为两个非负整数,且不同时为0;
    \item q为0时,则p为最大公约数;
    \item p \% q 余数为r;如果r为0,则结果为q;
    \item 如果r不为0;则p = q;q = r;继续上一步骤(\textbf{与余数辗转相除})
  \end{enumerate}

  \inputminted{java}{src/chapter01/GCD.java}

  \subsection{抽象数据类型(ADT)}
  抽象数据类型(Abstract Data Type,ADT),对算法结构的抽象,简化描述抽象的算法。

  \subsection{二分查找}
  \begin{enumerate}
    \item 已经排序的元素列表;以及待查询的目标元素;查询时,每次减半;
    \item 列表中间元素大于目标元素,则继续查询\textbf{[low, mid - 1]};
    \item 列表中间元素小于目标元素,则继续查询\textbf{[mid + 1, high]};
    \item 直到中间元素与目标元素相等,返回mid;或者low > high,返回-1。
  \end{enumerate}

  \inputminted{java}{src/chapter01/BinarySearch.java}

  \subsection{算术表达式运算}
  目前支持+、-、*、/运算符,参考:\href{http://faculty.cs.niu.edu/~hutchins/csci241/eval.htm}{算法描述} 。
  \begin{enumerate}
    \item 数值入栈,左括号入栈;
    \item 操作符号,如果操作符栈不为空,操作符栈顶符号不为左括号,且栈顶符号优先级 $\geq$ 当前扫描的符号,循环计算双栈;每次将结果压入操作数栈;最后压入当前操作符到栈中;
    \item 右括号,循环计算双栈,直到操作符栈为空,或达到左符号;每次计算的结果重新压入操作数栈;
    \item 最后操作符栈不为空,继续循环计算。
  \end{enumerate}

  \inputminted{java}{src/chapter01/ArithmeticExpression.java}

  \subsection{定容栈}
  固定大小的栈;如果需要扩容,提供resize方法,在push方法中检测是否需要扩容,然后调用resize方法;可以创建更大的数组,然后复制到新的数组中。

  \inputminted{java}{src/chapter01/FixedCapacityStack.java}

  \subsection{栈的链表实现}
  链表实现的栈没有限制容量。

  \inputminted{java}{src/chapter01/LinkedStack.java}

  \subsection{队列的链表实现}

  \inputminted{java}{src/chapter01/LinkedQueue.java}

  \subsection{背包}
  背包和栈类似,把push替换成add,然后移除pop;背包只添加元素,以及遍历元素。栈和队列也可以实现Iterator接口。

  \inputminted{java}{src/chapter01/Bag.java}

  \subsection{算法分析}
  \subsubsection{增长数量级分类}

  \begin{enumerate}
    \item 常数级别:$1$
    \item 对数级别:$\log N$
    \item 线性级别:$N$
    \item 线性对数级别:$N * \log N$
    \item 平方级别:$N^2$
    \item 对数级别:$N^3$
    \item 指数级别:$2^N$
  \end{enumerate}

  \subsubsection{2-sum}
  \inputminted{java}{src/chapter01/TwoSum.java}

  \subsubsection{3-sum}
  \inputminted{java}{src/chapter01/ThreeSum.java}

  \subsubsection{下界}
  在2-sum和3-sum例子中,是否能找到更优的算法,为算法在最坏的情况下的运行时间给出一个下界。

  \subsubsection{倍率实验}
  每次增加一倍输入规模,然后计算问题解决所需的时间,以此来预测任意问题的运行时间的增长数量级。

  \subsection{union-find算法}
  \subsubsection{动态连通性}
  输入一系列无序对,一对无序对p和q的关系为对等关系,即相连:
  \begin{enumerate}
    \item 自反性:p和p是相连,q和q是相连;
    \item 对称性:如果p和q相连,那么q和p相连;
    \item 传递性:如果p和q相连,q和r相连,那么p和r相连;
  \end{enumerate}
  对等关系能够将对象分为多个等价类;对象p和q相连时,那么p和q为同一分类。

  \subsubsection{Quick-Find实现}
  find()常数级别;union()线性级别,如果N个连通分量全部合并,则为平方级别。
  \inputminted{java}{src/chapter01/QuickFind.java}

  \subsubsection{Quick-Union实现}
  find()最坏情况下线性级别;union()最坏情况下线性级别,如果N个连通分量全部合并,则为平方级别;
  Quick-Union实现方式不一定比Quick-Find方式快,这取决于输入样本,取决于树的高度。
  \inputminted{java}{src/chapter01/QuickUnion.java}

  \subsubsection{加权Quick-Union实现}
  find()、union()、connected()最坏情况下也是对数级别。
  \inputminted{java}{src/chapter01/WeightedQuickUnion.java}

\end{document}
