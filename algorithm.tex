\documentclass{article}

\usepackage{geometry}
\usepackage{xeCJK}
\usepackage{minted}
\usepackage[colorlinks,linkcolor=blue]{hyperref}
\usepackage{amsmath}

% 设置页大小和页边距,或者scale=0.8
\geometry{a4paper,left=3.18cm,right=3.18cm,top=2.54cm,bottom=2.54cm}
% 声明分区:等宽字体,制表符号,等宽空格,ASCII
\xeCJKDeclareSubCJKBlock{MonoFont}{ "2500 -> "257F, "2002, "01 -> "F0 }
% 中文默认没有斜体和粗体格式,开启伪斜体和指定黑体;MonoFont分区指定字体为等宽字体Menlo
\setCJKmainfont[AutoFakeSlant, BoldFont={SimHei}, MonoFont=Menlo]{SimSun}

\begin{document}
\section{基础}
  \subsection{最大公约数}
  辗转相除法:
  \begin{enumerate}
    \item p和q为两个非负整数,且不同时为0;
    \item q为0时,则p为最大公约数;
    \item p \% q 余数为r;如果r为0,则结果为q;
    \item 如果r不为0;则p = q;q = r;继续上一步骤(\textbf{与余数辗转相除})
  \end{enumerate}

  \begin{minted}[linenos]{java}
  package chapter01;

  public class GCD {

      public static void main(String[] args) {
          System.out.println(gcd(18, 48));
          System.out.println(gcd(48, 18));
      }

      /**
       * find two number's greatest common divisor.
       * @param p
       * @param q
       * @return  greatest common divisor
       */
      private static int gcd(int p, int q) {
          if (q == 0) {
              return p;
          }

          int remainder;
          while ((remainder = p % q) != 0) {
              p = q;
              q = remainder;
          }

          return q;
      }

      //    递归版本
      //    private static int gcd(int p, int q) {
      //        if (q == 0) return p;
      //        int remainder = p % q;
      //        return gcd(q, remainder);
      //    }
  }
  \end{minted}

  \subsection{抽象数据类型(ADT)}
  抽象数据类型(Abstract Data Type,ADT),对算法结构的抽象,简化描述抽象的算法。

  \subsection{二分查找}
  \begin{enumerate}
    \item 已经排序的元素列表;以及待查询的目标元素;查询时,每次减半;
    \item 列表中间元素大于目标元素,则继续查询\textbf{[low, mid - 1]};
    \item 列表中间元素小于目标元素,则继续查询\textbf{[mid + 1, high]};
    \item 直到中间元素与目标元素相等,返回mid;或者low > high,返回-1。
  \end{enumerate}

  \begin{minted}[linenos]{java}
  package chapter01;

  public class BinarySearch {

      public static void main(String[] args) {
          int[] sortedArr1 = {2, 4, 6, 8, 9};
          int[] sortedArr2 = {1, 3, 5, 7, 9};
          System.out.println(binarySearch(4, sortedArr1));
          System.out.println(binarySearch(3, sortedArr2));
      }

      /**
       * 二分查询
       * @param target    待查询目标元素
       * @param sortedArr 升序数组
       * @return          目标元素下标,-1为不存在
       */
      private static int binarySearch(int target, int[] sortedArr) {
          int low = 0;
          int high = sortedArr.length - 1;

          while (low <= high) {
              int mid = low + (high - low) / 2;
              if (sortedArr[mid] < target) {
                  low = mid + 1;
              } else if (sortedArr[mid] > target) {
                  high = mid - 1;
              } else {
                  return mid;
              }
          }

          return -1;
      }
  }
  \end{minted}

  \subsection{算术表达式运算}
  目前支持+、-、*、/运算符,参考:\href{http://faculty.cs.niu.edu/~hutchins/csci241/eval.htm}{算法描述} 。
  \begin{enumerate}
    \item 数值入栈,左括号入栈;
    \item 操作符号,如果操作符栈不为空,操作符栈顶符号不为左括号,且栈顶符号优先级 $\geq$ 当前扫描的符号,循环计算双栈;每次将结果压入操作数栈;最后压入当前操作符到栈中;
    \item 右括号,循环计算双栈,直到操作符栈为空,或达到左符号;每次计算的结果重新压入操作数栈;
    \item 最后操作符栈不为空,继续循环计算。
  \end{enumerate}

  \begin{minted}[linenos, fontsize=\small]{java}
  package chapter01;

  import java.util.Stack;

  public class ArithmeticExpression {

      public static void main(String[] args) {
          double result1 = calcExpression("1 + ((2 - 4) * (12 / 3)) - 10");
          double result2 = calcExpression("3 - (6 - (10 - 1))");
          System.out.println(result1);
          System.out.println(result2);
      }

      /**
       * 栈顶部的操作符优先级 >= 当前扫描的操作符优先级?
       * ┌───┬────────────┐
       * │   │ +  -  *  / │
       * ├───┼────────────┤
       * │ + │ T  T  F  F │
       * │ - │ T  T  F  F │
       * │ * │ T  T  T  T │
       * │ / │ T  T  T  T │
       * └───┴────────────┘
       */
      private static boolean[][] opCompare = {
              {true, true, false, false},
              {true, true, false, false},
              {true, true, true, true},
              {true, true, true, true}
      };

      /**
       * 计算表达式
       * @param expression  算术表达式
       * @return            结果
       */
      private static double calcExpression(String expression) {
          if (expression == null || expression.isEmpty()) {
              throw new IllegalArgumentException("\"expression\" can not be empty.");
          }
          // 操作数
          Stack<Double> operands = new Stack<Double>();
          // 操作符
          Stack<Character> operators = new Stack<Character>();
          // 扫描表达式
          for (int i = 0; i < expression.length(); i++) {
              // 数字压栈
              if (isNum(expression.charAt(i))) {
                  int numStart = i;
                  while (++i < expression.length() && isNum(expression.charAt(i))) {}
                  operands.push(Double.parseDouble(expression.substring(numStart, i--)));
                  continue;
              }
              switch (expression.charAt(i)) {
                  case '(':
                      operators.push(expression.charAt(i));
                      break;
                  case '+':
                  case '*':
                  case '-':
                  case '/':
                      // 栈中的操作符号优先级高或同级别,则可以直接计算结果
                      // 运算是从左到右的,因此要及时计算结果
                      int rowIndex = getOpIndex(operators.peek());
                      int colIndex = getOpIndex(expression.charAt(i));
                      while (!operators.isEmpty()
                             && operators.peek() != '('
                             && opCompare[rowIndex][colIndex]) {
                          operands.push(calc(operands.pop(), operands.pop(), operators.pop()));
                      }
                      operators.push(expression.charAt(i));
                      break;
                  case ')':
                      // 计算括号内的表达式
                      char operator;
                      while (!operators.empty() && (operator = operators.pop()) != '(') {
                          operands.push(calc(operands.pop(), operands.pop(), operator));
                      }
                      break;
                  default:
              }
          }
          while (!operators.isEmpty()) {
              operands.push(calc(operands.pop(), operands.pop(), operators.pop()));
          }
          // 计算整个表达式
          return operands.isEmpty() ? Double.NaN : operands.pop();
      }

      /**
       * 判断字符是否是数字
       * @param c 字符
       * @return  true - 数字;false - 非数字
       */
      private static boolean isNum(char c) {
          return c >= '0' && c <= '9';
      }

      /**
       * 获取操作符优先级表opCompare的下标
       * @param op  操作符
       * @return    下标
       */
      private static int getOpIndex(char op) {
          switch (op) {
              case '+':
                  return 0;
              case '-':
                  return 1;
              case '*':
                  return 2;
              case '/':
                  return 3;
              default:
                  return -1;
          }
      }

      /**
       * 计算(operand2 op operand1)的值
       * @param operand1  右操作数
       * @param operand2  左操作数
       * @param op        操作符
       * @return          运算结果
       */
      private static double calc(double operand1, double operand2, char op) {
          switch (op) {
              case '+':
                  return operand2 + operand1;
              case '-':
                  return operand2 - operand1;
              case '*':
                  return operand2 * operand1;
              case '/':
                  return operand2 / operand1;
              default:
                  return Double.NaN;
          }
      }
  }
  \end{minted}

  \subsection{定容栈}
  固定大小的栈。

  \begin{minted}[linenos]{java}
  package chapter01;

  import java.util.EmptyStackException;

  public class FixedCapacityStack<T> {

      public static void main(String[] args) {
          FixedCapacityStack<String> stack = new FixedCapacityStack<>(5);
          System.out.println("stack empty: " + stack.isEmpty());
          for (int i = 0; i < 5; i++) {
              stack.push(i + 1 + "");
              System.out.println("stack size: " + stack.size());
              System.out.println("stack peek: " + stack.peek());
          }
          try {
              stack.push("6");
          } catch (StackOverflowError e) {
              System.out.println("stack overflow error.");
          }
          for (int i = 0; i < 5; i++) {
              System.out.println("stack pop: " + stack.pop());
          }
          try {
              stack.pop();
          } catch (EmptyStackException e) {
              System.out.println("stack empty error.");
          }
      }

      /**
       * element collection
       */
      private Object[] data;

      /**
       * element number
       */
      private int size = 0;

      /**
       * Fixed capacity
       * @param maxSize   max capacity
       */
      public FixedCapacityStack(int maxSize) {
          data = new Object[maxSize];
      }

      /**
       * add element
       * @param item  element
       */
      public void push(T item) {
          if (size == data.length) {
              throw new StackOverflowError();
          }
          data[size++] = item;
      }

      /**
       * get element from top stack
       * @return  element
       */
      public T pop() {
          if (isEmpty()) {
              throw new EmptyStackException();
          }
          T obj = peek();
          // to let gc do its work.
          data[--size] = null;
          return obj;
      }

      /**
       * see element from top stack
       * @return  element
       */
      @SuppressWarnings("unchecked")
      public T peek() {
          if (isEmpty()) {
              throw new EmptyStackException();
          }
          return (T) data[size - 1];
      }

      public boolean isEmpty() {
          return size == 0;
      }

      public int size() {
          return size;
      }
  }
  \end{minted}

\end{document}
